\section{Experimental environment}
\label{sec:system configs}

All experiments were carried out in one of three distinct machines that will be referred to as \textbf{Alpha}, \textbf{Bravo} and \textbf{Charlie}.
Their CPU and GPU hardware configurations are described in Tables \ref{tab:alpha}, \ref{tab:bravo} and \ref{tab:charlie}, respectively.
Besides, Charlie has a \emph{Seagate ST2000DM001} 7200 RPM spinning disk and a \emph{Samsung 840 EVO} Solid State Drive, informations relevant for the third phase of EAC.% with sequential read/write speeds of up to 540/410 MB/s.

Software wise, all machines are running Linux based operating systems.
Alpha and Bravo are using the Ubuntu 14.04 and 12.04, respectively, with a graphical interface.
Whether the machine is running a graphical interface or not is important because the amount of time a GPU kernel can take when the GPU is being used as a display device is severely constrained. %, in the case that it only has one GPU (as is the case with all the machines here presented) the available memory for computation is less than total and there is a limit to how long a CUDA kernel can be executed.
Charlie is running Fedora 21 without a user interface.

% SAMSUNG HM641JI
% http://www.farnell.com/datasheets/841934.pdf
% 5400 RPM
% 209.3IOPS


% Mighty4 SSD : http://hexus.net/tech/reviews/storage/58229-samsung-ssd-840-evo-120gb/
% needs booktabs package
% table inside minipage to have footnotes at the end
% system configuration of Samsung RV520.
\begin{minipage}[h]{\hsize}
	\centering
	\captionof{table}{\textbf{Alpha} machine specifications.}
	\begin{tabular}{ccc}
		\toprule[2pt]
												 & \textbf{CPU}      & \textbf{GPU} \\ \midrule
		\textbf{\# Devices}                      & 1                 & 1            \\ 
		\textbf{Manufacturer}                    & Intel             & NVIDIA       \\ 
		\textbf{Model}                           & i3-2310M          & GT 520M      \\ 
		\textbf{Launch date}                     & Q1'11             & Q1'2011      \\ 
		\textbf{Architecture}                    & Sandy Bridge      & Fermi        \\ 
		\textbf{\# Cores}                        & 2                 & 48           \\ 
		\textbf{Clock frequency {[}Mhz{]}}       & 2100              & 1480         \\ 
		\textbf{L1 Cache}                        & 64KB IC \footnote{Instruction Cache (IC)} + 64KB DC \footnote{Data Cache (IC)} & 16/48 KB/SM \footnote{Each Streaming Multiprocessor has 64 KB of on-chip memory that can be configured as either 16KB of L1 cache and 48 KB of shared memory, or vice versa.}  \\ 
		\textbf{L2 Cache}                        & 512KB             & n/a          \\ 
		\textbf{L3 Cache}                        & 3 MB              & n/a          \\ 
		\textbf{Memory {[}GB{]}}                 & 4                 & 1            \\ 
		\textbf{Max. memory bandwidth {[}Gbps{]}} & 21.3              & 12.8        \\ 
		\bottomrule[2pt]
	\end{tabular}
	\label{tab:alpha}
\end{minipage}
% needs booktabs package
% table inside minipage to have footnotes at the end
% Mighty4 configuration an Instituto de Telecomunicações at 04OCT2015
\begin{minipage}[h]{\hsize}
	\centering
	\captionof{table}{\textbf{Bravo} machine specifications.}
	\begin{tabular}{ccc}
		\toprule[2pt]
												 & \textbf{CPU}      & \textbf{GPU} \\ \midrule
		\textbf{\# Devices}                      & 1                 & 1            \\ 
		\textbf{Manufacturer}                    & Intel             & NVIDIA       \\ 
		\textbf{Model}                           & i7-4930K          & Quadro K600      \\ 
		\textbf{Launch date}                     & Q3'13             & Q1'2013      \\ 
		\textbf{Architecture}                    & Ivy Bridge      &         \\ 
		\textbf{\# Cores}                        & 6                 & 192           \\ 
		\textbf{Clock frequency {[}Mhz{]}}       & 3400              & 876         \\ 
		\textbf{L1 Cache}                        & 192 KB IC \footnote{Instruction Cache (IC)} + 192 KB DC \footnote{Data Cache (IC)} & 16/48 KB/SM \footnote{Each Streaming Multiprocessor has 64 KB of on-chip memory that can be configured as either 16KB of L1 cache and 48 KB of shared memory, or vice versa.} + 48KB DC \footnote{The Kepler architecture has an extra read-only 48KB of Data Cache at the same level of the L1 cache.} \\ 
		\textbf{L2 Cache}                        & 1,5 MB             & 1.5 MB         \\ 
		\textbf{L3 Cache}                        & 12 MB              & n/a          \\ 
		\textbf{Memory {[}GB{]}}                 & 32                & 1            \\ 
		\textbf{Max. memory bandwidth {[}Gbps{]}} & 59,6              & 28,5        \\ 
		\bottomrule[2pt]
	\end{tabular}
	\label{tab:bravo}
\end{minipage}
% needs booktabs package
% table inside minipage to have footnotes at the end
% Mariana configuration an INESC-ID at 04OCT2015
\begin{minipage}[h]{\hsize}
	\centering
	\captionof{table}{\textbf{Bravo} machine specifications.}
	\begin{tabular}{ccc}
		\toprule[2pt]
												 & \textbf{CPU}      & \textbf{GPU} \\ \midrule
		\textbf{\# Devices}                      & 1                 & 1            \\ 
		\textbf{Manufacturer}                    & Intel             & NVIDIA       \\ 
		\textbf{Model}                           & i7 4770K          & K40c      \\ 
		\textbf{Launch date}                     & Q2'13             & Q4'13      \\ 
		\textbf{Architecture}                    & Haswell      & Kepler        \\ 
		\textbf{\# Cores}                        & 4                 & 2880           \\ 
		\textbf{Clock frequency {[}Mhz{]}}       & 3500              & 745         \\ 
		\textbf{L1 Cache}                        & 128 KB IC \footnote{Instruction Cache (IC)} + 128 KB DC \footnote{Data Cache (IC)} & 16/48 KB/SM \footnote{Each Streaming Multiprocessor has 64 KB of on-chip memory that can be configured as either 16KB of L1 cache and 48 KB of shared memory, or vice versa.} + 48KB DC \footnote{The Kepler architecture has an extra read-only 48KB of Data Cache at the same level of the L1 cache.} \\ 
		\textbf{L2 Cache}                        & 1 MB             & 1.5 MB         \\ 
		\textbf{L3 Cache}                        & 8 MB              & n/a          \\ 
		\textbf{Memory {[}GB{]}}                 & 32                & 12            \\ 
		\textbf{Max. memory bandwidth {[}Gbps{]}} & 25,6              & 288        \\ 
		\bottomrule[2pt]
	\end{tabular}
	\label{tab:bravo}
\end{minipage}