%!TEX root = ExtendedAbstract.tex

%%%%%%%%%%%%%%%%%%%%%%%%%%%%%%%%%%%%%%%%%%%%%%%%%%%%%%%%%%%%%%%%%%%%%%
%     File: ExtendedAbstract_imple.tex                               %
%     Tex Master: ExtendedAbstract.tex                               %
%                                                                    %
%     Author: Andre Calado Marta                                     %
%     Last modified : 27 Dez 2011                                    %
%%%%%%%%%%%%%%%%%%%%%%%%%%%%%%%%%%%%%%%%%%%%%%%%%%%%%%%%%%%%%%%%%%%%%%
% A Calculation section represents a practical development
% from a theoretical basis.
%%%%%%%%%%%%%%%%%%%%%%%%%%%%%%%%%%%%%%%%%%%%%%%%%%%%%%%%%%%%%%%%%%%%%%

\section{Optimization of the combination step of EAC}
\label{sec:imple}

Space complexity is the main challenge building the co-association matrix.
A complete pair-wire matrix has $O(n^2)$ complexity but can be reduced to $O(\frac{n(n-1)}/2)$ without loss of information.
Still, these complexities are rather high when large data sets are contemplated, since it becomes impossible to fit these "conventional" co-association matrices in main memory.
Two solutions in literature address this challenge.
\cite{Fred2005} approaches it by using a $k$-Nearest Neighbor approach, only considering associations between the $k$ closes neighbors of each pattern.
\cite{Lourenco2010} approaches the problem by exploiting the sparse nature of the co-association matrix for reducing space complexity, but doesn't cover the efficiency of building a sparse matrix or the space overhead associated with sparse data structures.
The effort of the present work was focused on further exploiting the sparse nature of EAC, building on previous insights and exploring the topics that literature has neglected so far.

Building a non-sparse matrix is easy and fast since the memory for the whole matrix is allocated and indexing the matrix is direct.
When using sparse matrices, neither is true.
In the specific case of EAC, there is no way to know what is the number of associations the co-association matrix will have which means it is not possible to pre-allocate the memory to the correct size of the matrix.
This translates in allocating the memory gradually which may result in fragmentation, depending on the implementation, and more complex data structures, which incurs significant computational overhead.
For building a matrix, the DOK (Dictionary of Keys) and LIL (List of Lists) formats are recommended in the documentation of the SciPy \cite{JonesSciPy} scientific computation library.
These were briefly tested on simple EAC problems and not only was their execution time several orders of magnitude higher than a traditional fully allocated matrix, but the overhead of the sparse data strucutres resulted in a space complexity higher than what would be needed to process large data sets.
For operating over a matrix, the documentation recommends converting from one of the previous formats to either CSR (Compressed Sparse Row) or CSC (Compressed Sparse Column).
Building with the CSR format had a low space complexity but the execution time was much higher than even one of the other sparse formats.

The bad performance of the above sparse formats combined with the fact that no relevant literature was found on efficiently building sparse matrices led to the design and implementation of a novel strategy for a CSR matrix specialized to the EAC context, the \textbf{EAC CSR}.



